\setauthor{Weissengruber Nina}
\section{Endpoints}
\subsection{Transaction}
\subsubsection{find all}
\textbf{GET} \emph{https://localhost:8080/api/transaction}

\textbf{Rückgabewert}
Zurückgegeben wird eine Liste von allen Transactionen als JSON
Format.

\subsubsection{find By Id}
\textbf{GET} \emph{https://localhost:8080/api/transaction/id/\{id\}}

\textbf{Parameter}
\begin{itemize}
    \item \emph{id}: eindeutige Nummer einer Transaction
\end{itemize}

\textbf{Rückgabewert}
Zurückgegeben wird das gefundene Transaction Objekt als JSON Format. Wenn keines mit der angegeben \emph{id} gefunden wird, so wird der
HTTP Error \emph{No Content (204)} zurückgegeben.

\subsubsection{create Transaction}
\textbf{POST} \emph{https://localhost:8080/api/transaction}

\textbf{Body}
\begin{itemize}
    \item Transaction Objekt im JSON Format
    \item UriInfo
\end{itemize}

\textbf{Aktion}
In der Datenbank wird das übergebene Transaction Objekt persistiert.

\textbf{Rückgabewert}
Zurückgegeben wird die URI, über welche das soeben erstellte Objekt über einen anderen Endpoint angefordert werden kann.

\subsubsection{get Transaction with code}
\textbf{GET} \emph{https://localhost:8080/api/transaction/code/\{code\}}

\textbf{Parameter}
\begin{itemize}
    \item \emph{code}: String mit dem Code einer Transaction
\end{itemize}

\textbf{Rückgabewert}
Zurückgegeben wird das gefundene Transaction Objekt als JSON Format. Wenn keines mit der angegeben \emph{id} gefunden wird, so wird der
HTTP Error \emph{No Content (204)} zurückgegeben.

\subsubsection{get Survey with Transaction}
\textbf{POST} \emph{https://localhost:8080/api/transaction/getSurvey}

\textbf{Body}
\begin{itemize}
    \item Transaction Objekt im JSON Format
\end{itemize}

\textbf{Aktion}
Es werden alle Survey Objekte mit einer referenz auf das übergebene Transaction Objekt gesucht.

\textbf{Rückgabewert}
Zurückgegeben wird eine Liste von Survey Objekten.

\subsubsection{update Transaction}
\textbf{POST} \emph{https://localhost:8080/api/transaction/\{id\}}

\textbf{Body}
\begin{itemize}
    \item id: eindeutige Nummer einer Transaction
    \item Transaction Objekt im JSON Format
\end{itemize}

\textbf{Aktion}
In der Datenbank wird das wird das bereits persistierte Objekt aktualisiert.

\textbf{Rückgabewert}
Zurückgegeben wird das persistierte Objekt. Wenn keines mit der angegeben id gefunden wird, so wird der HTTP Error Bad Request
(400) zurückgegeben.

\subsubsection{delete Transaction}
\textbf{DELETE} \emph{https://localhost:8080/api/transaction/\{id\}}

\textbf{Body}
\begin{itemize}
    \item id: eindeutige Nummer einer Transaction
\end{itemize}

\textbf{Aktion}
Es wird das Transaction Objekt mit der übergebenen Id gesucht und anschließend
aus der Datenbank gelöscht.

\textbf{Rückgabewert}
Nach einem erfolgreichen lösch Vorgang wird der http status \emph{ok (200)} zurückgegeben.

\subsubsection{generate Transaction}
\textbf{GET} \emph{https://localhost:8080/api/transaction/generate/\{number\}/\{surveyId\}}

\textbf{Parameter}
\begin{itemize}
    \item \emph{surveyId}: eindeutige Nummer einer Survey
    \item \emph{number}: Anzahl wie viele Transactions generiert werdern müssen
\end{itemize}

\textbf{Aktion}
Es wird die Anzahl der gewünschten Transactions erstellt und in der Datenbank persistiert 

\textbf{Rückgabewert}
Zurückgegeben wird ein File in dem alle Codes der neu generierten Transactions stehen.

\subsection{Survey}
\subsubsection{find All}
\textbf{GET} \emph{https://localhost:8080/api/survey}

\textbf{Rückgabewert}
Zurückgegeben wird eine Liste von allen Survey Objekten als JSON
Format.

\subsubsection{find By Id}
\textbf{GET} \emph{https://localhost:8080/api/survey/id/\{id\}}

\textbf{Parameter}
\begin{itemize}
    \item \emph{id}: eindeutige Nummer einer Survey
\end{itemize}

\textbf{Rückgabewert}
Zurückgegeben wird das gefundene Survey Objekt als JSON Format. Wenn keines mit der angegeben \emph{id} gefunden wird, so wird der
HTTP Error \emph{No Content (204)} zurückgegeben.


\subsubsection{create Survey}
\textbf{POST} \emph{https://localhost:8080/api/survey}

\textbf{Body}
\begin{itemize}
    \item Survey Objekt im JSON Format
    \item UriInfo
\end{itemize}

\textbf{Aktion}
In der Datenbank wird das übergebene Survey Objekt persistiert.

\textbf{Rückgabewert}
Zurückgegeben wird die URI, über welche das soeben erstellte Objekt über einen anderen Endpoint angefordert werden kann.

\subsubsection{find Survey by Interviewer Id}
\textbf{GET} \emph{https://localhost:8080/api/survey/\{interviewerId\}}

\textbf{Parameter}
\begin{itemize}
    \item \emph{interviewerId}: eindeutige Nummer eines Interviewers
\end{itemize}

\textbf{Rückgabewert}
Zurückgegeben wird das gefundene Survey Objekt als JSON Format. 

\subsubsection{find Questionnaire with Survey Id}
\textbf{GET} \emph{https://localhost:8080/api/survey/questionnaire/\{surveyId\}}

\textbf{Parameter}
\begin{itemize}
    \item \emph{surveyId}: eindeutige Nummer einer Survey
\end{itemize}

\textbf{Rückgabewert}
Zurückgegeben wird das gefundene Questionnaire Objekt als JSON Format.

\subsubsection{update Survey}
\textbf{POST} \emph{https://localhost:8080/api/survey/\{id\}}

\textbf{Body}
\begin{itemize}
    \item id: eindeutige Nummer einer Survey
    \item Survey Objekt im JSON Format
\end{itemize}

\textbf{Aktion}
In der Datenbank wird das wird das bereits persistierte Objekt aktualisiert.

\textbf{Rückgabewert}
Zurückgegeben wird das persistierte Objekt. Wenn keines mit der angegeben id gefunden wird, so wird der HTTP Error Bad Request
(400) zurückgegeben.

\subsubsection{delete Survey}
\textbf{DELETE} \emph{https://localhost:8080/api/survey/\{id\}}

\textbf{Body}
\begin{itemize}
    \item id: eindeutige Nummer einer Survey
\end{itemize}

\textbf{Aktion}
Es wird das Survey Objekt mit der übergebenen Id gesucht und anschließend
aus der Datenbank gelöscht.

\textbf{Rückgabewert}
Nach einem erfolgreichen lösch Vorgang wird der http status \emph{ok (200)} zurückgegeben.

\subsubsection{Auswertung}
\textbf{GET} \emph{https://localhost:8080/api/survey/evaluation/\{surveyId\}}

\textbf{Parameter}
\begin{itemize}
    \item \emph{surveyId}: eindeutige Nummer einer Survey
\end{itemize}

\textbf{Rückgabewert}
Zurückgegeben wird ein File mit der Auswertung.

\subsection{Question}
\subsubsection{find All}
\textbf{GET} \emph{https://localhost:8080/api/question}

\textbf{Rückgabewert}
Zurückgegeben wird eine Liste von allen Question Objekten als JSON
Format.

\subsubsection{find By Id}
\textbf{GET} \emph{https://localhost:8080/api/question/id/\{id\}}

\textbf{Parameter}
\begin{itemize}
    \item \emph{id}: eindeutige Nummer einer Question
\end{itemize}

\textbf{Rückgabewert}
Zurückgegeben wird das gefundene Question Objekt als JSON Format. Wenn keines mit der angegeben \emph{id} gefunden wird, so wird der
HTTP Error \emph{No Content (204)} zurückgegeben.

\subsubsection{create Question}
\textbf{POST} \emph{https://localhost:8080/api/question}

\textbf{Body}
\begin{itemize}
    \item Question Objekt im JSON Format
    \item UriInfo
\end{itemize}

\textbf{Aktion}
In der Datenbank wird das übergebene Question Objekt persistiert.

\textbf{Rückgabewert}
Zurückgegeben wird die URI, über welche das soeben erstellte Objekt über einen anderen Endpoint angefordert werden kann.

\subsubsection{update Question}
\textbf{POST} \emph{https://localhost:8080/api/question/\{id\}}

\textbf{Body}
\begin{itemize}
    \item id: eindeutige Nummer einer Question
    \item Question Objekt im JSON Format
\end{itemize}

\textbf{Aktion}
In der Datenbank wird das wird das bereits persistierte Objekt aktualisiert.

\textbf{Rückgabewert}
Zurückgegeben wird das persistierte Objekt. Wenn keines mit der angegeben id gefunden wird, so wird der HTTP Error Bad Request
(400) zurückgegeben.

\subsubsection{delete Question}
\textbf{DELETE} \emph{https://localhost:8080/api/question/\{id\}}

\textbf{Body}
\begin{itemize}
    \item id: eindeutige Nummer einer Question
\end{itemize}

\textbf{Aktion}
Es wird das Question Objekt mit der übergebenen Id gesucht und anschließend
aus der Datenbank gelöscht.

\textbf{Rückgabewert}
Nach einem erfolgreichen lösch Vorgang wird der http status \emph{ok (200)} zurückgegeben.

\subsection{Questionnaire}
\subsubsection{find All}
\textbf{GET} \emph{https://localhost:8080/api/questionnaire}

\textbf{Rückgabewert}
Zurückgegeben wird eine Liste von allen Questionnaire Objekten als JSON
Format.

\subsubsection{find By Id}
\textbf{GET} \emph{https://localhost:8080/api/questionnaire/id/\{id\}}

\textbf{Parameter}
\begin{itemize}
    \item \emph{id}: eindeutige Nummer einer Questionnaire
\end{itemize}

\textbf{Rückgabewert}
Zurückgegeben wird das gefundene Questionnaire Objekt als JSON Format. Wenn keines mit der angegeben \emph{id} gefunden wird, so wird der
HTTP Error \emph{No Content (204)} zurückgegeben.

\subsubsection{create Questionnaire}
\textbf{POST} \emph{https://localhost:8080/api/questionnaire}

\textbf{Body}
\begin{itemize}
    \item Questionnaire Objekt im JSON Format
    \item UriInfo
\end{itemize}

\textbf{Aktion}
In der Datenbank wird das übergebene Questionnaire Objekt persistiert.

\textbf{Rückgabewert}
Zurückgegeben wird die URI, über welche das soeben erstellte Objekt über einen anderen Endpoint angefordert werden kann.

\subsubsection{find public Questionnaires}
\textbf{GET} \emph{https://localhost:8080/api/questionnaire/public}

\textbf{Rückgabewert}
Zurückgegeben wird eine Liste von Questionnaire Objekten, bei denen das Attribut isPublic auf true gestezt ist.

\subsubsection{get public and owned Questionnaires}
\textbf{GET} \emph{https://localhost:8080/api/questionnaire/public/\{interviewerId\}}

\textbf{Body}
\begin{itemize}
    \item interviewerId: eindeutige Nummer eines Interviewers
\end{itemize}

\textbf{Rückgabewert}
Zurückgegeben wird eine Liste von Questionnaire Objekten, bei denen das Attribut isPublic auf true gestezt ist 
und eine referenz auf das übergebene Interviewer Onjekt besteht.

\subsubsection{find Questionnaire By Interviwer id}
\textbf{GET} \emph{https://localhost:8080/api/questionnaire/interviwer/\{interviewerId\}}

\textbf{Body}
\begin{itemize}
    \item interviewerId: eindeutige Nummer eines Interviewers
\end{itemize}

\textbf{Rückgabewert}
Zurückgegeben wird eine Liste von Questionnaire Objekten, bei denen 
eine referenz auf das übergebene Interviewer Onjekt besteht.

\subsubsection{duplicate Questionnaire}
\textbf{POST} \emph{https://localhost:8080/api/duplicateQuestionnaire}

\textbf{Body}
\begin{itemize}
    \item Questionnaire Objekt im JSON Format
\end{itemize}

\textbf{Aktion}
Es wird ein duplicat des übergebenen Questionnaire Objekts in der Datenbank persistiert.

\textbf{Rückgabewert}
Zurückgegeben wird das duplizierte Questionnaire Objekt.

\subsubsection{update Questionnaire}
\textbf{POST} \emph{https://localhost:8080/api/questionnaire/\{id\}}

\textbf{Body}
\begin{itemize}
    \item id: eindeutige Nummer einer Questionnaire
    \item Questionnaire Objekt im JSON Format
\end{itemize}

\textbf{Aktion}
In der Datenbank wird das wird das bereits persistierte Objekt aktualisiert.

\textbf{Rückgabewert}
Zurückgegeben wird das persistierte Objekt. Wenn keines mit der angegeben id gefunden wird, so wird der HTTP Error Bad Request
(400) zurückgegeben.

\subsubsection{delete Questionnaire}
\textbf{DELETE} \emph{https://localhost:8080/api/questionnaire/\{id\}}

\textbf{Body}
\begin{itemize}
    \item id: eindeutige Nummer eines Questionnaire
\end{itemize}

\textbf{Aktion}
Es wird das Questionnaire Objekt mit der übergebenen Id gesucht und anschließend
aus der Datenbank gelöscht.

\textbf{Rückgabewert}
Nach einem erfolgreichen lösch Vorgang wird der http status \emph{ok (200)} zurückgegeben.

\subsection{Interviewer}
\subsubsection{find All}
\textbf{GET} \emph{https://localhost:8080/api/interviewer}

\textbf{Rückgabewert}
Zurückgegeben wird eine Liste von allen Interviewer Objekten als JSON
Format.

\subsubsection{find By Id}
\textbf{GET} \emph{https://localhost:8080/api/interviewer/id/\{id\}}

\textbf{Parameter}
\begin{itemize}
    \item \emph{id}: eindeutige Nummer eines Interviewers
\end{itemize}

\textbf{Rückgabewert}
Zurückgegeben wird das gefundene Interviewer Objekt als JSON Format. Wenn keines mit der angegeben \emph{id} gefunden wird, so wird der
HTTP Error \emph{No Content (204)} zurückgegeben.

\subsubsection{create Question}
\textbf{POST} \emph{https://localhost:8080/api/interviewer}

\textbf{Body}
\begin{itemize}
    \item Interviewer Objekt im JSON Format
    \item UriInfo
\end{itemize}

\textbf{Aktion}
In der Datenbank wird das übergebene Interviewer Objekt persistiert.

\textbf{Rückgabewert}
Zurückgegeben wird die URI, über welche das soeben erstellte Objekt über einen anderen Endpoint angefordert werden kann.

\subsubsection{update Interviewer}
\textbf{POST} \emph{https://localhost:8080/api/interviewer/\{id\}}

\textbf{Body}
\begin{itemize}
    \item id: eindeutige Nummer einer Interviewer
    \item Interviewer Objekt im JSON Format
\end{itemize}

\textbf{Aktion}
In der Datenbank wird das wird das bereits persistierte Objekt aktualisiert.

\textbf{Rückgabewert}
Zurückgegeben wird das persistierte Objekt. Wenn keines mit der angegeben id gefunden wird, so wird der HTTP Error Bad Request
(400) zurückgegeben.

\subsubsection{delete Interviewer}
\textbf{DELETE} \emph{https://localhost:8080/api/interviewer/\{id\}}

\textbf{Body}
\begin{itemize}
    \item id: eindeutige Nummer einer Interviewer
\end{itemize}

\textbf{Aktion}
Es wird das Interviewer Objekt mit der übergebenen Id gesucht und anschließend
aus der Datenbank gelöscht.

\textbf{Rückgabewert}
Nach einem erfolgreichen lösch Vorgang wird der http status \emph{ok (200)} zurückgegeben.

\subsection{ChosenOption}
\subsubsection{find All}
\textbf{GET} \emph{https://localhost:8080/api/chosenOption}

\textbf{Rückgabewert}
Zurückgegeben wird eine Liste von allen ChosenOption Objekten als JSON
Format.

\subsubsection{find By Id}
\textbf{GET} \emph{https://localhost:8080/api/chosenOption/id/\{id\}}

\textbf{Parameter}
\begin{itemize}
    \item \emph{id}: eindeutige Nummer einer ChosenOption
\end{itemize}

\textbf{Rückgabewert}
Zurückgegeben wird das gefundene ChosenOption Objekt als JSON Format. Wenn keines mit der angegeben \emph{id} gefunden wird, so wird der
HTTP Error \emph{No Content (204)} zurückgegeben.

\subsubsection{create ChosenOption}
\textbf{POST} \emph{https://localhost:8080/api/chosenOption}

\textbf{Body}
\begin{itemize}
    \item ChosenOption Objekt im JSON Format
    \item UriInfo
\end{itemize}

\textbf{Aktion}
In der Datenbank wird das übergebene ChosenOption Objekt persistiert.

\textbf{Rückgabewert}
Zurückgegeben wird die URI, über welche das soeben erstellte Objekt über einen anderen Endpoint angefordert werden kann.

\subsubsection{update ChosenOption}
\textbf{POST} \emph{https://localhost:8080/api/chosenOption/\{id\}}

\textbf{Body}
\begin{itemize}
    \item id: eindeutige Nummer einer ChosenOption
    \item ChosenOption Objekt im JSON Format
\end{itemize}

\textbf{Aktion}
In der Datenbank wird das wird das bereits persistierte Objekt aktualisiert.

\textbf{Rückgabewert}
Zurückgegeben wird das persistierte Objekt. Wenn keines mit der angegeben id gefunden wird, so wird der HTTP Error Bad Request
(400) zurückgegeben.

\subsubsection{delete ChosenOption}
\textbf{DELETE} \emph{https://localhost:8080/api/chosenOption/\{id\}}

\textbf{Body}
\begin{itemize}
    \item id: eindeutige Nummer einer ChosenOption
\end{itemize}

\textbf{Aktion}
Es wird das ChosenOption Objekt mit der übergebenen Id gesucht und anschließend
aus der Datenbank gelöscht.

\textbf{Rückgabewert}
Nach einem erfolgreichen lösch Vorgang wird der http status \emph{ok (200)} zurückgegeben.

\subsection{Answer}
\subsubsection{find All}
\textbf{GET} \emph{https://localhost:8080/api/answer}

\textbf{Rückgabewert}
Zurückgegeben wird eine Liste von allen Answer Objekten als JSON
Format.

\subsubsection{find By Id}
\textbf{GET} \emph{https://localhost:8080/api/answer/id/\{id\}}

\textbf{Parameter}
\begin{itemize}
    \item \emph{id}: eindeutige Nummer einer Answer
\end{itemize}

\textbf{Rückgabewert}
Zurückgegeben wird das gefundene Answer Objekt als JSON Format. Wenn keines mit der angegeben \emph{id} gefunden wird, so wird der
HTTP Error \emph{No Content (204)} zurückgegeben.

\subsubsection{create Answer}
\textbf{POST} \emph{https://localhost:8080/api/answer}

\textbf{Body}
\begin{itemize}
    \item Answer Objekt im JSON Format
    \item UriInfo
\end{itemize}

\textbf{Aktion}
In der Datenbank wird das übergebene Answer Objekt persistiert.

\textbf{Rückgabewert}
Zurückgegeben wird die URI, über welche das soeben erstellte Objekt über einen anderen Endpoint angefordert werden kann.

\subsubsection{update Answer}
\textbf{POST} \emph{https://localhost:8080/api/answer/\{id\}}

\textbf{Body}
\begin{itemize}
    \item id: eindeutige Nummer einer Answer
    \item Answer Objekt im JSON Format
\end{itemize}

\textbf{Aktion}
In der Datenbank wird das wird das bereits persistierte Objekt aktualisiert.

\textbf{Rückgabewert}
Zurückgegeben wird das persistierte Objekt. Wenn keines mit der angegeben id gefunden wird, so wird der HTTP Error Bad Request
(400) zurückgegeben.

\subsubsection{delete Answer}
\textbf{DELETE} \emph{https://localhost:8080/api/answer/\{id\}}

\textbf{Body}
\begin{itemize}
    \item id: eindeutige Nummer einer Answer
\end{itemize}

\textbf{Aktion}
Es wird das Answer Objekt mit der übergebenen Id gesucht und anschließend
aus der Datenbank gelöscht.

\textbf{Rückgabewert}
Nach einem erfolgreichen lösch Vorgang wird der http status \emph{ok (200)} zurückgegeben.

\subsection{AnswerOption}
\subsubsection{find All}
\textbf{GET} \emph{https://localhost:8080/api/answerOption}

\textbf{Rückgabewert}
Zurückgegeben wird eine Liste von allen AnswerOption Objekten als JSON
Format.

\subsubsection{find By Id}
\textbf{GET} \emph{https://localhost:8080/api/answerOption/id/\{id\}}

\textbf{Parameter}
\begin{itemize}
    \item \emph{id}: eindeutige Nummer einer AnswerOption
\end{itemize}

\textbf{Rückgabewert}
Zurückgegeben wird das gefundene AnswerOption Objekt als JSON Format. Wenn keines mit der angegeben \emph{id} gefunden wird, so wird der
HTTP Error \emph{No Content (204)} zurückgegeben.

\subsubsection{create Answer}
\textbf{POST} \emph{https://localhost:8080/api/answerOption}

\textbf{Body}
\begin{itemize}
    \item AnswerOption Objekt im JSON Format
    \item UriInfo
\end{itemize}

\textbf{Aktion}
In der Datenbank wird das übergebene AnswerOption Objekt persistiert.

\textbf{Rückgabewert}
Zurückgegeben wird die URI, über welche das soeben erstellte Objekt über einen anderen Endpoint angefordert werden kann.

\subsubsection{update AnswerOption}
\textbf{POST} \emph{https://localhost:8080/api/answerOption/\{id\}}

\textbf{Body}
\begin{itemize}
    \item id: eindeutige Nummer einer AnswerOption
    \item AnswerOption Objekt im JSON Format
\end{itemize}

\textbf{Aktion}
In der Datenbank wird das wird das bereits persistierte Objekt aktualisiert.

\textbf{Rückgabewert}
Zurückgegeben wird das persistierte Objekt. Wenn keines mit der angegeben id gefunden wird, so wird der HTTP Error Bad Request
(400) zurückgegeben.

\subsubsection{delete AnswerOption}
\textbf{DELETE} \emph{https://localhost:8080/api/answerOption/\{id\}}

\textbf{Body}
\begin{itemize}
    \item id: eindeutige Nummer einer AnswerOption
\end{itemize}

\textbf{Aktion}
Es wird das AnswerOption Objekt mit der übergebenen Id gesucht und anschließend
aus der Datenbank gelöscht.

\textbf{Rückgabewert}
Nach einem erfolgreichen lösch Vorgang wird der http status \emph{ok (200)} zurückgegeben.

